\documentclass{beamer}
%\documentclass[aspectratio=169]{beamer}
%
\mode<presentation>
{
  \usetheme{default}      
  \usecolortheme{default}
  \usefonttheme{default} 
  \setbeamertemplate{navigation symbols}{}
  \setbeamertemplate{caption}[numbered]
} 

\usepackage[english]{babel}
\usepackage[utf8x]{inputenc}
\usepackage{bbm}
\usepackage{bm}

\newcommand{\1}[1]{\mathbbm{1}\left[#1\right]}
\newcommand{\norm}[1]{\left\lVert#1\right\rVert}
\newcommand{\yi}{y^{(i)}}
\newcommand{\yhat}{\hat{y}}
\newcommand{\yhati}{\hat{y}^{(i)}}
\newcommand{\bx}{\bm{x}}
\newcommand{\bxi}{\bx^{(i)}}
\newcommand{\pv}{\pause\vfill}
\newcommand{\p}{\pause}
\newcommand{\f}{\vfill}

\title{Machine learning from scratch}
\subtitle{Lecture 7: Classification}
\author{Alexis Zubiolo\newline\texttt{alexis.zubiolo@gmail.com}}
\institute{Data Science Team Lead @ Adcash}
\date{March 16, 2017}

\begin{document}

\begin{frame}
  \titlepage
\end{frame}

\begin{frame}{Introduction}
\textbf{Classification} and \textbf{regression} are the 2 main aspects of supervised machine learning.
\pv
\textbf{Reminder}: In classification, $y$ takes \textbf{discrete values}: \textbf{1 value per class}. Typically, we note $y \in \{1, 2, \dots, k \}$ where $k$ is the number of classes.
\pv
\textbf{Example}: 
\begin{itemize}
	\item For digit recognition, there are \textbf{10 classes} (one per digit: 0, 1, 2, \ldots, 9)
	\item For letter recognition, the number of classes depends on the alphabet (e.g. 26 for the latin alphabet).
\end{itemize}
\end{frame}

\begin{frame}{Support Vector Machines}
\textbf{Support Vector Machines} (SVM) is one of the most popular classification algorithms. What we will see in this course is just one way to approach it. Actually, it can be solved/approached in several (yet equivalent) ways.
\pv
As usual, we will define a dataset $\left\{ (\bxi, \yi), i =1, \dots, n\right\}$. Here, $\yi \in \{ 1, \dots, k\}$ is the class.
\pv
One way to formulate the SVM model is by defining \textbf{one linear models per class}: $\theta_1, \theta_2, \dots, \theta_k$. The goal is to have $\theta_j^T x$ measure the confidence of $\bx$ having $j$ as a label. This way, the predicted class of $\bx$ will be the $j$ that maximizes $\theta^T \bx$.
\end{frame}

\begin{frame}{Support Vector Machines}
Let us consider a single sample $(\bx, y)$. The SVM algorithm will define weights $\theta_1, \theta_2, \dots, \theta_k$ for each class.
\pv
Hence, we can define scores $s_j = \theta_j^T \bx$ for each class $j$. If the SVM algorithm makes no mistake on the sample $(\bx, y)$, $s_y$ \textbf{should be the highest value}.
\end{frame}

\begin{frame}{Hinge loss}
SVM relies on the \textbf{hinge loss}, that is
\begin{equation*}
\ell(s_j, s_y) = \max(0, s_j - s_y + \Delta)
\end{equation*}
where:
\begin{itemize}
	\item $s_j$ is the score for some class $j \neq y$
	\item $s_y$ is the score for the true class $y$
	\item $\Delta$ is a \textbf{hyper-parameter} that quantifies by how much we want $s_y$ to be bigger than $s_j$ for $j \neq y$
\end{itemize}
\pv
This loss has to be computed for all the classes \textbf{other than the actual class }$y$. In the end, the total loss for the sample $(\bx, y)$ is the sum of the loss for all the $j \neq y$:
\begin{equation*}
L = \sum_{j \neq y} \ell(s_j, s_y) = \sum_{j \neq y} \max(0, s_j - s_y + \Delta)
\end{equation*}
\end{frame}

\begin{frame}{Illustration on a simple example}
\textbf{Example}: Suppose we have 3 classes (1, 2 and 3). We get a training sample $(\bx, y)$ within the first class (that is, $y = 0$). We have an SVM algorithm with hyperparameter $\Delta = 10$ that returns scores of 13 for class 1, −7 for class 2, and 11 for class 3. 

\textbf{Question}: What is the total loss for this sample?
\pv
\textbf{Solution}: 
\begin{equation*}
\begin{split}
L 	&=  \ell(s_2, s_1) + \ell(s_3, s_1) \\
	&= \max(0,−7−13+10) + \max(0,11−13+10) \\
	&= \max(0, -10) + \max(0, +8) \\
	&= 0 + 8 \\
	&= 8
\end{split}
\end{equation*}
\end{frame}

\begin{frame}{Optimization problem formulation}
We saw that for a single sample $(\bx, y)$, the training loss is:
\begin{equation*}
L = \sum_{j \neq y} \ell(s_j, s_y) = \sum_{j \neq y} \max(0, s_j - s_y + \Delta)
\end{equation*}
where $s_j = \theta_j^T \bx$ and $s_y = \theta_y^T \bx$.
\pv
As for the linear regression, when given a training set $\left\{ (\bxi, \yi), i =1, \dots, n\right\}$, we just need to sum the losses for all the training samples:
\begin{equation*}
\begin{split}
J(\theta_1, \theta_2, \dots, \theta_k) 
	&= \sum_{i = 1}^n L_i \\
	&= \sum_{i = 1}^n \sum_{j \neq y_i} \max(0, \theta_j^T\bxi - \theta_{y_i}^T\bxi + \Delta)
\end{split}
\end{equation*}
\end{frame}

\begin{frame}{Overfitting and regularization}
As for regression tasks, \textbf{outliers} can have a dramatic impact on the model we train. We saw that it usually resulted in high weights and that \textbf{adding a regularization term} could help.
\begin{equation*}
R(\theta_1, \theta_2, \dots, \theta_k) = \sum_{j = 1}^k \norm{\theta_j}_2^2
\end{equation*}
\pv
We can do the same for the SVM problem:
\begin{equation*}
J(\theta_1, \theta_2, \dots, \theta_k) 
	= \sum_{i = 1}^n \sum_{j \neq y_i} 
	\max(0, \theta_j^T\bxi - \theta_{y_i}^T\bxi + \Delta)
	+ \lambda \sum_{j = 1}^k \norm{\theta_j}_2^2
\end{equation*}
where
\begin{equation*}
\norm{\theta_j}_2^2 = \sum_{m = 1}^d \theta_{j,m}^2
\end{equation*}
($\theta_j$ is a vector, and $\theta_{j,m}$ its $m$th value.)
\end{frame}

\begin{frame}{Optimization}
We have defined a cost function $J$ we would like to minimize. Again, we can apply an algorithm like gradient descent to find its optimal value. For this, we need to compute $J$'s \textbf{gradients}.
\pv
In this case, there will be \textbf{several gradients} (1 per class because we have 1 $\theta_j$ vector per class). We will note $\nabla_{\theta_j}J$ the gradient with respect to $\theta_j$.
\pv
Due to the gradient linearity, we can compute the gradient of each component \textbf{separately}. Hence, we need to compute:
\begin{itemize}
	\item $L$'s gradients
	\item $R$'s gradients
\end{itemize}
\end{frame}

\begin{frame}{$L$'s gradients}
\textbf{Reminder}: For a sample $(\bx, y)$, $L$ is given by:
\begin{equation*}
L = \sum_{j \neq y} \max(0, \theta_j^T\bx - \theta_{y}^T\bx + \Delta)
\end{equation*}
\pv
There are 2 different cases:
\pv
\begin{itemize}
	\item $j \neq y$:
	\begin{equation*}
	\nabla_{\theta_j} L = 
	\begin{cases}
	\bx \quad \text{if} \quad \theta_j^T\bx - \theta_{y}^T\bx + \Delta > 0 \\
	0 \quad \text{otherwise.}
	\end{cases}
	\end{equation*}
\pv
	\item $j = y$:
	\begin{equation*}
	\nabla_{\theta_j} L = p \bx
	\end{equation*}
	where $p$ is the \textbf{number of times the desired margin is not satisfied}, that is the number of $j \neq y$ such that 
	$$\theta_j^T\bx - \theta_{y}^T\bx + \Delta > 0$$
\end{itemize}
\end{frame}

\begin{frame}{$R$'s gradients}
\textbf{Reminder}: $R$ is given by:
\begin{equation*}
R(\theta_1, \theta_2, \dots, \theta_k) = \sum_{j = 1}^k \norm{\theta_j}_2^2
\end{equation*}
\textbf{Question}: What is $\nabla_{\theta_j} R$?
\pv
\textbf{Answer}:
\begin{equation*}
\nabla_{\theta_j} R = 2\theta_j
\end{equation*}
\pv
\textbf{Conclusion}: We have the analytic expression of the gradients of $L$ and $R$ and are able to apply the gradient descent (stochastic or batch) to it.
\end{frame}

\begin{frame}{Conclusion}
In this lecture, we've seen Classification models with the example of SVMs:
\begin{itemize}
	\item How to define the problem
	\item How to compute the gradient
	\item How to train an SVM model for multiclass classification
\end{itemize}
\pv
Next time, we'll go a bit further and have a \textbf{practical session} about classification.
\end{frame}

\begin{frame}
\begin{center}
\Huge{Thank you! Questions?}
\end{center}
\end{frame}

\end{document}