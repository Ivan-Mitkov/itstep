\documentclass{beamer}
%\documentclass[aspectratio=169]{beamer}
%
\mode<presentation>
{
  \usetheme{default}      
  \usecolortheme{default}
  \usefonttheme{default} 
  \setbeamertemplate{navigation symbols}{}
  \setbeamertemplate{caption}[numbered]
} 

\usepackage[english]{babel}
\usepackage[utf8x]{inputenc}
\usepackage{bbm}
\usepackage{bm}

\newcommand{\1}[1]{\mathbbm{1}\left[#1\right]}
\newcommand{\norm}[1]{\left\lVert#1\right\rVert}
\newcommand{\yi}{y^{(i)}}
\newcommand{\yhat}{\hat{y}}
\newcommand{\yhati}{\hat{y}^{(i)}}
\newcommand{\bx}{\bm{x}}

\title{Machine learning from scratch}
\subtitle{Lecture 7: Classification}
\author{Alexis Zubiolo\newline\texttt{alexis.zubiolo@gmail.com}}
\institute{Data Science Team Lead @ Adcash}
\date{March 16, 2017}

\begin{document}

\begin{frame}
  \titlepage
\end{frame}

\begin{frame}{Extension to multiclass classification}
\vfill
What we have seen so far works for \textit{binary classification} (2 classes). What if we have \textbf{3 classes or more}?
\vfill
\pause 
Several possible extensions. Some of the most popular ones:
\begin{itemize}
	\item One against one classification
	\item One against all classification
\end{itemize}
\vfill
\textbf{Note}: These strategies apply to any binary classifier (including SVMs).
\vfill
\end{frame}
%
\begin{frame}{One against one classification}
\textbf{Idea}: 
\begin{itemize}
	\item Compute a classifier \textbf{for all pairs} of classes.
	\item \textbf{Apply all these classifiers} to the new point. Store the predicted classes.
	\item \textbf{Majority vote}: Return the class with the highest number of votes
\end{itemize}

\end{frame}
%
\begin{frame}{One against all classification}
\vfill
\textbf{Note}: It is also called \textit{one against rest classification}.
\vfill
\textbf{Idea}: For each class, split the set of classes into two meta classes
\begin{itemize}
	\item The considered class
	\item The union of all the other classes
\end{itemize}
and compute a classifier for all these possibilities.
Apply all these classifiers to a (new) given point.
\vfill
\textbf{Final decision} based on the value of $\theta^T x$.
\end{frame}

\begin{frame}
\begin{center}
\Huge{Thank you! Questions?}
\end{center}
\end{frame}

\end{document}